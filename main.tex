\documentclass[12pt]{article}
\usepackage[utf8]{inputenc}
% \loadspellchecklist[pl]
\usepackage{hyperref}
\usepackage{polski}
\usepackage[polish]{babel}
\usepackage{adjustbox}
\usepackage{multirow}
\usepackage{booktabs}
\usepackage{hhline}
\usepackage[table,xcdraw]{xcolor}
\usepackage{array}
\usepackage{boldline}
\usepackage{transparent}
\usepackage{tabularx}
\usepackage{longtable}
\usepackage{caption}
\usepackage{fancyhdr}
\usepackage[left=2cm,top=3.9cm,right=2cm,bottom=2cm,headsep=2cm]{geometry}
\pagestyle{fancy}
\usepackage{tikz}
\usetikzlibrary{calc}
\usepackage{tikzpagenodes}
\fancyhead{} % clear all header fields
\renewcommand{\headrulewidth}{0pt}
\lhead{\begin{tikzpicture}[remember picture,overlay]
{\transparent{0.5}
\includegraphics[width=8cm]{logo PWr kolor poziom.png}}
\end{tikzpicture}}

\rhead{
{\transparent{0.6}
 Wydział Elektroniki\\
Katedra Telekomunikacji i Teleinformatyki\\}}
\fancyfoot{}
\fancyfoot[R]{Strona \thepage}

\definecolor{redd}{HTML}{88100d}

\begin{document}
% \thispagestyle{empty} % title page style without headers or footers
\begin{table}[ht!]
\begin{adjustbox}{height=\textheight, width=\textwidth, keepaspectratio}
\begin{tabular}{|lcllcV{5}}
\hlineB{6}
\multicolumn{1}{V{5}l|}{Specjalność:}                   & \multicolumn{3}{l|}{\cellcolor[HTML]{FFFFC7}Nazwa zajęć:}                         &                                                                                                \\
\multicolumn{1}{V{5}c|}{}                               & \multicolumn{3}{c|}{\cellcolor[HTML]{FFFFC7}}                                     &                                                                                                \\
\multicolumn{1}{V{5}c|}{}                               & \multicolumn{3}{c|}{\cellcolor[HTML]{FFFFC7}}                                     &                                                                                                \\
\multicolumn{1}{V{5}c|}{\multirow{-3}{*}{\Large\textbf{TSI}}} & \multicolumn{3}{c|}{\multirow{-3}{*}{\cellcolor[HTML]{FFFFC7}\Large \color{redd}\textbf{LABORATORIUM ANTEN}}} & \multirow{-4}{*}{\begin{tabular}[c]{@{}c@{}}TERMIN\\ Wtorek\\ \large\textbf{P/N}\\ 7:30 - 11:00\end{tabular}} \\ \hlineB{6}
\multicolumn{4}{V{5}l|}{Osoby wykonujące ćwiczenie:}                                                                                        & \multicolumn{1}{lV{5}}{Nr Grupy}                                                                  \\
\multicolumn{4}{V{5}l|}{\begin{tabular}[c]{@{}l@{}}\hspace*{2cm}1. Pan/Pani X\\ \hspace*{2cm}2. Pan/Pani Y\\ \hspace*{2cm}3. Pan/Pani Z\end{tabular}}                              & \Large\textbf{5}                                                                                     \\ \hlineB{6}
\multicolumn{4}{V{5}l|}{Tytuł ćwiczenia:}                                                                                                   & \multicolumn{1}{lV{5}}{Nr ćwiczenia}                                                              \\
\multicolumn{4}{V{5}c|}{}                                                                                                                   &                                                                                                \\
\multicolumn{4}{V{5}c|}{}                                                                                                                   &                                                                                                \\
\multicolumn{4}{V{5}c|}{\multirow{-3}{*}{\begin{tabular}[c]{@{}c@{}}Dobór zysku energetycznego anteny odbiorczej w oparciu\\ 
o pomiary poziomu sygnału użytecznego, obliczenia propagacyjne \\
oraz bilans łącza radiowego.
\\ \end{tabular}}}                                            & \multirow{-3}{*}{\Large\textbf{2}}                                                                   \\ \hlineB{6}
\multicolumn{2}{V{5}l|}{Sprawozdanie opracował}                                     & \multicolumn{2}{lV{5}}{Pan/Pani X}                       & \cellcolor[HTML]{FFFFC7}OCENA                                                                  \\ \cline{1-4}
\multicolumn{2}{V{5}l|}{Data wykonania ćwiczenia}                                   & \multicolumn{2}{lV{5}}{25 kwietnia 2017}                 & \multicolumn{1}{lV{5}}{\cellcolor[HTML]{FFFFC7}}                                                  \\ \cline{1-4}
\multicolumn{2}{V{5}l|}{Data oddania sprawozdania}                                  & \multicolumn{2}{lV{5}}{09 maja 2017}                     & \multicolumn{1}{lV{5}}{\cellcolor[HTML]{FFFFC7}}                                                  \\ \cline{1-4}
\multicolumn{2}{V{5}l|}{Prowadzący}                                                 & \multicolumn{2}{lV{5}}{dr hab. inż. Piotr Słobodzian}    & \multicolumn{1}{lV{5}}{\multirow{-3}{*}{\cellcolor[HTML]{FFFFC7}}}                                \\ \hlineB{6}
\multicolumn{5}{V{5}lV{5}}{\begin{tabular}[c]{@{}l@{}}\color{redd}\textbf{UWAGI} (wypełnia prowadzący):\\ \\ \\ \\ \\ \\ \\ \\ \\ \\ \\ \\ \\ \\ \\ \\  1\end{tabular}}                                                                                                              \\ \hlineB{6}
\end{tabular}
\end{adjustbox}
\end{table}
\newpage
\section{First section}
Type your document as usual!
\newpage
\section{Second section}
Type your document as usual!

\end{document}
